\documentclass[12pt,nonacm,natbib=false]{acmart}
\settopmatter{printfolios=false,printccs=false,printacmref=false}
\usepackage[maxnames=1,minnames=1,natbib=true,citestyle=authoryear,bibstyle=authoryear,doi=false,url=false,isbn=false,isbn=false]{biblatex}
\makeatletter\let\@authorsaddresses\@empty\makeatother
\usepackage[T1]{fontenc}
\usepackage[tt=false,type1=true]{libertine}
\usepackage{amsmath}
\usepackage{multicol}
\usepackage{ffcode}
\usepackage{tabularx}
\usepackage{href-ul}
\usepackage{xcolor}
\usepackage{paralist}
\usepackage{microtype}
\title{Academic Writing in \LaTeX: \texorpdfstring{\\}{} Best and Worst Practices}
\author{Yegor Bugayenko}
\addbibresource{main.bib}
\pagenumbering{gobble}
\raggedbottom
\setlength{\parindent}{0pt}
\setlength{\columnsep}{32pt}
\setlength{\parskip}{6pt}
\setlength{\headheight}{18pt}
\setlength{\footskip}{18pt}

\begin{abstract}
    This is a humble attempt to summarize most typical mistakes we make while writing academic papers in \LaTeX{} and most important recommendations. Each suggestion or a mistake takes a short paragraph of description right here and also may suggest looking into a more detailed explanation in some other online resource. We recommend, before submitting your paper to a conference or a journal you go through this list of mistakes and make sure none of them are present in your paper.
\end{abstract}

\begin{document}
\pagestyle{plain}
\date{}
\maketitle

\LaTeX{} sources of this document you can find in
\href{https://github.com/yegor256/latex-best-practices}{this GitHub repository}
and contribute your ideas through a pull request.

Beforehand, we suggest you read these:
\begin{itemize}
\item \href{https://developers.google.com/tech-writing/overview}{Technical Writing Courses by Google}
\end{itemize}

\section{Preamble}

Use \href{https://ctan.org/pkg/acmart?lang=en}{\ff{acmart}} document style and read their \href{https://www.acm.org/publications/taps/latex-best-practices}{Best Practices}. Start the document with this:

\begin{ffcode}
\documentclass[11pt,nonacm,natbib=false]{acmart}
\settopmatter{printfolios=false,printccs=false,printacmref=false}
\usepackage[maxnames=1,minnames=1,natbib=true,
  citestyle=authoryear,bibstyle=authoryear]{biblatex}
\addbibresource{main.bib}
\end{ffcode}

Use \href{https://ctan.org/pkg/biblatex}{\ff{biblatex}} and \href{https://ctan.org/pkg/biber}{\ff{biber}}, here is \href{https://tex.stackexchange.com/a/25702/1449}{why}. Place your citations into \ff{main.bib} file. Later in the document print the bibliography with \ff{\char`\\printbibliography} command.

\section{Headings}

Capitalize all nouns and verbs in headings, \href{https://apastyle.apa.org/style-grammar-guidelines/capitalization/title-case}{here} is why and how.

\section{Typography}

Use single dash inside words, e.g.: \ff{micro-service}.
Use double ``endash'' between numbers, e.g.: \ff{pages 15-{}-28}.
Use triple ``emdash'' between words avoiding spaces, e.g.: \ff{We-{}-{}-since you ask-{}-{}-disagree}.
Read \href{https://tex.stackexchange.com/questions/3819/dashes-vs-vs}{this}.

\section{Fonts}

Prefer \ff{\char`\\emph} over \ff{\char`\\textit}, here is \href{https://tex.stackexchange.com/a/1983/1449}{why}.

Avoid \ff{\char`\\textbf} and all other font changing commands at all cost. Here is \href{https://www.yegor256.com/2019/05/21/dont-improvise.html}{my rant} on this very problem of technical people trying to make their products look visually attractive and failing miserably.

\section{Colors}

Don't use them. Keep your documents strictly black-on-white. Read \href{https://academia.stackexchange.com/questions/13616/are-there-good-reasons-to-avoid-using-color-in-research-papers}{more} about this.

\section{Code Snippets}

Use \href{https://ctan.org/pkg/ffcode}{\ff{ffcode}} package, which allows writing both code snippets and fixed-width-font in-paragraph text blocks.

\section{Figures and Tables}

Don't force positioning in figures and tables, like \ff{\char`\\begin\{table\}[h]}. Instead, just \ff{\char`\\begin\{table\}}.

Make sure the explanation you place into \ff{\char`\\caption} is detailed enough to let your reader understand the content without searching the text; see how it's done in \href{https://arxiv.org/abs/2203.07814}{this paper}.

Prefer a list over a table and a table over a graph.

Align text cells by left, center headings, and align cells with numbers by right; \href{https://ux.stackexchange.com/questions/24066/what-is-the-best-practice-for-data-table-cell-content-alignment}{here} is a more detailed discussion. Here is an example of a table properly formatted:

\begin{multicols}{2}
\setlength{\parskip}{0pt}
\small
\raggedcolumns
\begin{verbatim}
\begin{tabularx}{\columnwidth}
  {lr>{\raggedright\arraybackslash}X}
\toprule
Name & Age & Role \\
\midrule
Jeff & 35 & The creator of the main
algorithm and the owner of the code \\
Sarah & 38 & The architect of all
microservices and the developer of
Java modules \\
\bottomrule
\end{tabularx}
\end{verbatim}

\columnbreak

\begin{tabularx}{\columnwidth}
  {lr>{\raggedright\arraybackslash}X}
\toprule
Name & Age & Role \\
\midrule
Jeff & 35 & The creator of the main algorithm and the owner of the code \\
Sarah & 38 & The architect of all microservices and the developer of Java modules \\
\bottomrule
\end{tabularx}
\end{multicols}

Put all tables into \ff{table} environment:

\begin{ffcode}
\begin{table}
.. content goes here
\caption{Caption}
\label{tab:my-table}
\end{table}
\end{ffcode}


\section{Bullets}

Prefer in-paragraph itemization over a vertical one and use \href{https://ctan.org/pkg/paralist}{\ff{paralist}}:

\begin{multicols}{2}
\setlength{\parskip}{0pt}
\small
\raggedcolumns
\begin{verbatim}
The following sources were analyzed:
\begin{inparaenum}[1)]
\item GitHub,
\item Google, and
\item Stack Overflow
\end{inparaenum}
\end{verbatim}

\columnbreak

\raggedright
The following sources were
analyzed:
\begin{inparaenum}[1)]
\item GitHub,
\item Google, and
\item Stack Overflow
\end{inparaenum}
\end{multicols}

In all itemization use \href{https://en.wikipedia.org/wiki/Serial_comma}{Oxford comma}, as in the list above before the ``and'' (provided there are more than two items).

\section{References}

Don't use \ff{\char`\\ref}, use \ff{\char`\\cref} instead from \href{https://ctan.org/pkg/cleveref}{\ff{cleveref}} package.

\section{Citations}

This code demonstrates how to use \href{https://libguides.murdoch.edu.au/APA}{APA}-style citations with \href{https://journals.aas.org/natbib/}{\ff{natbib}} commands:

\begin{multicols}{2}
\setlength{\parskip}{0pt}
\small
\raggedcolumns
\begin{verbatim}
In \citeyear{west2004} it was
already mentioned
by \citeauthor{west2004}
that object-oriented design is
declarative~\citep{west2004}.
Later, \citet{eolang2021}
suggested a new programming
language in this paradigm.
\end{verbatim}

\columnbreak

\raggedright
In~\citeyear{west2004} it was already
mentioned by~\citeauthor{west2004} that
object-oriented design is
declarative~\citep{west2004}.
Later, \citet{eolang2021}
suggested a new programming
language in this paradigm.

\AtNextBibliography{\small}
\setlength\bibitemsep{3pt}
{\raggedright\printbibliography[heading=none]}
\end{multicols}

Place \ff{\~} (tilde) symbol before \ff{\char`\\citep} to avoid line breaks, \href{https://tex.stackexchange.com/questions/41264/what-is-the-difference-in-citing-referencing-with-or-without-tilde}{see why}.

Prefer \ff{\char`\\citet} over \ff{\char`\\citep}, making references more obvious, as in the second sentence in the example above.

Don't type author names or reference titles directly, only use \ff{\char`\\cite*} commands.

\section{References}

The references in \ff{.bib} file are usually imported from Google Scholar or similar sources. Unfortunately, such imports often contain typos and mistakes. Check the items printed in the ``References'' section for the following:

\begin{itemize}
    \item Year is not printed twice;
    \item Dashes in titles are printed as \ff{-{}-{}-} without surrounding spaces;
    \item All nouns and verbs are capitalized in all titles.
\end{itemize}

Use \href{https://github.com/Kingsford-Group/biblint}{biblint} to check your \ff{.bib} file.

\end{document}
